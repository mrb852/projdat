\documentclass[]{article}

\begin{document}

\title{Projektkursus Delrapport 1.}
\author{Christian Enevoldsen, Simon Warg, Nicklas Warming, Robert Rasmussen}
\date{\today}
\maketitle

\section*{Indledning}

Igennem et gruppemedlem, fik vi kontakt til IT og webhosting firmaet Surftown A/S. Surftown er et såkaldt web hosting firma, som tilbyder tjenster i form af domæner, web hosting og webdesign. En typisk tjenestepakke vil f.eks. bestå af et domæne samt hosting plads på en delt server, hvor kunden af mulighed for at uploade og/eller udvikle deres website. For at holde udgifterne (og derved produktprisen) nede samt gøre det nemt for kunden, har Surftowns kunder mulighed for at logge på et personligt kontrolpanel. I dette kontrolpanel har Surftowns kunder adgang til flere forskellige værktøjer så som; FTP adgang, database håndtering, faktura oversigt, bestiling af domæner osv.

Udviklingen i dag går i mod at man kan tilgå og administrere de vigtigste ting på farten, via en smartphone eller lgn. Derfor er det også en naturlig forventning at Suftowns kunder kan tilgå og administrere de vigtigste tjeneste hos Surftown, denne forventning opfylder Suftown imidlertidig ikke.

Da Surftown A/S er et stort firma med titusindvis af kunder i både Danmark, Sverige og Norger, samt nogle få øvrige lande, er projektet utroligt interresant da vi i tilfælde af success, kan få feedback fra et stort antal bruger, fra en relativ bred vifte af lande. Herudover er det en chance for at interagerer og få indsigt i et veletableret forma med et existrende og velfungerende IT infrastruktur.

\section*{Problemformulering}

PROBLEM FORMULERING:
Surftowns kunder har mulighed for at administrer deres hosting tjenester igennem Surftown webbaseret kontrolpanel. Surftowns webbaseret kontrolpanel er dog ikke optimeret til at blive vist på mindre skærme, hvilket har den konsekvens at surftowns kunder ikke har mulighed for at få information eller administrere deres hosting tjeneste, medmindre de har adgang til en traditionel computer med internetforbindelse. Da det er tvivlsomt at Surftowns kunder vil lave større ændringer på deres hosting tjenester mens de ikke har adgang til en traditionel computer, ønsker Surftown hovedsagtligt en informativ tjeneste, som er mulig at tilgå mobilt. De ønskede og aftalte tjeneste er som følger:


\begin{enumerate}
\item Surftowns kontaktoplysninger.
\item En guide til at sætte Surftowns e-mail oplysninger op på en smartphone.
\item Mulighed for at recover sit bruger password.
\item En login form.
\end{enumerate}

\textbf{\\Naar brugeren er logget ind aktiveres disse tjenester\\}


\begin{enumerate}
\item En oversigt over brugerens domæner.
\begin{itemize}
\item  Status for domænet.
\item En oversigt over brugerens fakturaer for domænet.
\item Dato for registrering af domænet.
\item Udløbningsdato for domænet.
\item Løbende pris for registreringen domænet.
\end {itemize}
\begin{enumerate} 
\item Oversigt over brugerens webhosting.
\begin{itemize}
\item Brugt/ledigt plads.
\item Oversigt over fakturaer for webhostingen.
\item Løbende pris for webhostingen.
\item OS type (linux/windows).
\item Antal tilladte tilknyttede domæner.
\end {itemize}
\begin{enumerate} 
\item Driftstatus.
\begin{itemize}
\item Webhosting.
\item Email.
\item Databaser.
\item Kontrolpanelet.
\item Webmail.
\end {itemize}
\begin{enumerate} 
\item Udvidet kontakt
\begin{itemize}
\item Oprette og læse support tickets.
\item Visning af tilkøbte priotets supportkoder.
\item Support videoer.
\end {itemize}
\end {enumerate}
\section*{Indledende skitse og plan}

\end{document}